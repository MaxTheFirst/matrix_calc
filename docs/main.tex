\documentclass[a4paper, 14pt]{article}
\usepackage[T2A]{fontenc}			      % кодировка
\usepackage[utf8]{inputenc}               % кодировка исходного текста
\usepackage[english, russian]{babel}   % локализация и переносы


%%% Страница 
\usepackage{extsizes} % Возможность сделать 14-й шрифт
\usepackage{geometry}  
\geometry{left=20mm,right=20mm,top=25mm,bottom=30mm} % задание полей текста


%%%  Текст
\setlength\parindent{0pt}         % Устанавливает длину красной строки 0pt
\sloppy                                        % строго соблюдать границы текста
\linespread{1.3}                           % коэффициент межстрочного интервала
\setlength{\parskip}{0.5em}                % вертик. интервал между абзацами
%\setcounter{secnumdepth}{0}                % отключение нумерации разделов
\usepackage{multicol}				          % Для текста в нескольких колонках
%\usepackage{soul}
\usepackage{soulutf8} % Модификаторы начертания


%%% Гиппер ссылки
\usepackage{hyperref}
\usepackage[usenames,dvipsnames,svgnames,table,rgb]{xcolor}
\hypersetup{				% Гиперссылки
	unicode=true,           % русские буквы в раздела PDF\\
	pdfstartview=FitH,
	pdftitle={Заголовок},   % Заголовок
	pdfauthor={Автор},      % Автор
	pdfsubject={Тема},      % Тема
	pdfcreator={Создатель}, % Создатель
	pdfproducer={Производитель}, % Производитель
	pdfkeywords={keyword1} {key2} {key3}, % Ключевые слова
	colorlinks=true,       	% false: ссылки в рамках; true: цветные ссылки
	linkcolor=blue,          % внутренние ссылки
	citecolor=green,        % на библиографию
	filecolor=magenta,      % на файлы
	urlcolor=NavyBlue,           % на URL
}


%%% Для формул
\usepackage{amsmath}          
\usepackage{amssymb}


\usepackage{amsthm}  % for theoremstyle

\theoremstyle{plain} % Это стиль по умолчанию, его можно не переопределять.
\newtheorem*{theorem}{Теорема}
\newtheorem*{prop}{Утверждение}
\newtheorem*{lemma}{Лемма}
\newtheorem*{sug}{Предположение}

\theoremstyle{definition} % "Определение"
\newtheorem*{Def}{Определение}
\newtheorem*{corollary}{Следствие}
\newtheorem{problem}{Задача}[section]

\theoremstyle{remark} % "Примечание"
\newtheorem*{nonum}{Решение}
\newtheorem*{defenition}{Def}
\newtheorem*{example}{Пример}
\newtheorem*{note}{Замечание}


%%% Работа с картинками
\usepackage{graphicx}                           % Для вставки рисунков
\graphicspath{{images/}{images2/}}        % папки с картинками
\setlength\fboxsep{3pt}                    % Отступ рамки \fbox{} от рисунка
\setlength\fboxrule{1pt}                    % Толщина линий рамки \fbox{}
\usepackage{wrapfig}                     % Обтекание рисунков текстом
\graphicspath{{images/}}                     % Путь к папке с картинками


%%% облегчение математических обозначений


\newcommand{\all}{\forall}
\newcommand{\ex}{\exists}

\newcommand{\RR}{\mathbb{R}}
\newcommand{\NN}{\mathbb{N}}
%\newcommand{\C}{\mathbb{C}}             % команда уже определена где-то)
\newcommand{\ZZ}{\mathbb{Z}}
\newcommand{\EE}{\mathbb{E}}
\newcommand{\brackets}[1]{\left({#1}\right)}      % автоматический размер скобочек
% Здесь можно добавить ваши индивидуальные сокращения  

\makeatletter
\def\@seccntformat#1{\@ifundefined{#1@cntformat}%
   {\csname the#1\endcsname\space}%    default
   {\csname #1@cntformat\endcsname}}%  enable individual control
\def\section@cntformat{\thesection.\space} % section-level
\makeatother


\title{Калькулятор для матриц.}
\date{}

\begin{document}
\maketitle
\section{UI.}
\begin{enumerate}
    \item Выбор действия (найти определитель, найти обратную). 
    \item Выбор порядка матрицы (2*2, 3*3).
    \item При выборе появляется соответствующая таблица полей для ввода. 
    \item Кнопка "Вычислить".
    \item (Опционально) окно предпросмотра Latex.
    \item Текстовое поле с кодом.
    \item Кнопка "Копировать".
\end{enumerate}
\section{Вычисление определителя.}
\texttt{2-й порядок} -- считаем по формуле, возвращаем одно значение. \\ 
\texttt{3-й порядок} -- раскладываем по первой строке, возвращаем миноры элементов первой строки, и само значение. \\
\section{Вычисление обратной.} 
Длаем $(A|E)$ и с помощью ЭП-строк приводим к $(E|A^{-1})$. Возвращаем матрицу на каждом этапе. \\ \\
При дробных значениях можно возвращать кортеж $(a, b)$. 
\section{Вывод в Latex.} 
\subsection{}
\[
    \begin{vmatrix}
        1&2 \\ 3&4
    \end{vmatrix}
    =1\cdot 4 - 2 \cdot 3=-2
\]
\subsection{}
\[
    \begin{vmatrix}
        1&2&3 \\ 4&5&6 \\ 7&8&9
    \end{vmatrix}
    =1 \cdot
    \begin{vmatrix}
        5&6 \\ 8&9
    \end{vmatrix}
    -2 \cdot
    \begin{vmatrix}
        4&6 \\ 7&9
    \end{vmatrix}
    +3 \cdot
    \begin{vmatrix}
        4&5 \\ 7&8
    \end{vmatrix}
    =0
\]
\subsection{}

\[
(A | E) = \begin{bmatrix}
1 & 2 & 3 & 1 & 0 & 0 \\
6 & 7 & 8 & 0 & 1 & 0 \\
11 & 12 & 13 & 0 & 0 & 1
\end{bmatrix}
\]

\[
\begin{bmatrix}
1 & 2 & 3 & 1 & 0 & 0 \\
0 & -5 & -10 & -6 & 1 & 0 \\
11 & 12 & 13 & 0 & 0 & 1
\end{bmatrix}
\]

\[
\begin{bmatrix}
1 & 2 & 3 & 1 & 0 & 0 \\
0 & -5 & -10 & -6 & 1 & 0 \\
0 & -10 & -28 & -11 & 0 & 1
\end{bmatrix}
\]

\[
\begin{bmatrix}
1 & 2 & 3 & 1 & 0 & 0 \\
0 & 1 & 2 & \frac{6}{5} & -\frac{1}{5} & 0 \\
0 & -10 & -28 & -11 & 0 & 1
\end{bmatrix}
\]

\[
\begin{bmatrix}
1 & 2 & 3 & 1 & 0 & 0 \\
0 & 1 & 2 & \frac{6}{5} & -\frac{1}{5} & 0 \\
0 & 0 & -8 & -1 & 2 & 1
\end{bmatrix}
\]

\[
\begin{bmatrix}
1 & 2 & 3 & 1 & 0 & 0 \\
0 & 1 & 2 & \frac{6}{5} & -\frac{1}{5} & 0 \\
0 & 0 & 1 & \frac{1}{8} & -\frac{1}{4} & -\frac{1}{8}
\end{bmatrix}
\]

\[
\begin{bmatrix}
1 & 2 & 3 & 1 & 0 & 0 \\
0 & 1 & 0 & \frac{43}{40} & \frac{3}{20} & \frac{1}{8} \\
0 & 0 & 1 & \frac{1}{8} & -\frac{1}{4} & -\frac{1}{8}
\end{bmatrix}
\]

\[
\begin{bmatrix}
1 & 0 & 0 & \frac{3}{4} & \frac{1}{2} & \frac{1}{4} \\
0 & 1 & 0 & \frac{43}{40} & \frac{3}{20} & \frac{1}{8} \\
0 & 0 & 1 & \frac{1}{8} & -\frac{1}{4} & -\frac{1}{8}
\end{bmatrix}
\]

\[
A^{-1} = \begin{bmatrix}
\frac{3}{4} & \frac{1}{2} & \frac{1}{4} \\
\frac{43}{40} & \frac{3}{20} & \frac{1}{8} \\
\frac{1}{8} & -\frac{1}{4} & -\frac{1}{8}
\end{bmatrix}
\]
\end{document}